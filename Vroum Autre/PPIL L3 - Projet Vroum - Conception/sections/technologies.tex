\chapter{Technologies}

\subsection*{PHP v7.3~: Langage de programmation pour back-end de site web}

\hangindent=\parindent%
Technologie très utilisée dans le monde du web, relativement pratique et ayant beaucoup de ressources et de bibliothèques.
          
\subsection*{Composer v2~: Gestionnaire de paquets pour PHP}
    
\hangindent=\parindent%
Permet de créer, maintenir et déployer une application très simplement en 3 commandes.

\hangindent=\parindent%
Permet aussi de gérer automatiquement les différents paquets requis pour bien développer l'application.

\subsection*{Eloquent v8~: Bibliothèque de gestion de base de données par association avec des objets}
    
\hangindent=\parindent%
Autrement appelée «~ORM~» (Object-Relational Mapping), cette bibliothèque permet en quelques lignes de définir un squelette de table de base de données dans une classe, et d'avoir une corrélation entre la base de données et les classes de l'application.
          
\subsection*{Slim v4~: Framework de routage}
        
\hangindent=\parindent%
Slim permet de gérer les URLs de façon propre et  plus facilement dans le code. Des redirections d'URLs sont également possibles.

\subsection*{JS ES6~: Langage de programmation utilisé pour le front-end d'un site web}
    
\hangindent=\parindent%
La partie back-end de JS (via Node.JS) ne nous intéresse pas ici.

\hangindent=\parindent%
JS permet d'avoir une interface assez dynamique, en complétant HTML et CSS.
          
\subsection*{jQuery v3.5~: Framework JS}
    
\hangindent=\parindent%
jQuery est un très gros frameworks JS permettant de faire beaucoup de choses natives à JS, mais de façon plus simple et plus ergonomique.

\hangindent=\parindent%
Ce framework est inclus dans Bootstrap jusqu'à la version 4.

\subsection*{Bootstrap v4~: Framework CSS}
          
\hangindent=\parindent%
Permet d'avoir des classes CSS prédéfinies pour rendre plus ergonomique l'application, et avoir un style joli sans passer 3 ans sur le CSS.
          
\subsection*{Gitlab~: Gestionnaire de code et de tâches collaboratif}
    
\hangindent=\parindent%
Gitlab permet de gérer du code de façon collaborative, entre différents utilisateurs.

\hangindent=\parindent%
Il permet aussi de gérer des tâches via le mécanisme d'issues, et son intégration d'un trello.

\hangindent=\parindent%
Cette plateforme est accessible via le VCS \texttt{git}, ou directement en ligne.
    
\subsection*{MySQL + MariaDB v10~: Système de gestion de base de données (SGBD)}
    
\hangindent=\parindent%
Permet d'assurer la persistance des données à relativement grande échelle.
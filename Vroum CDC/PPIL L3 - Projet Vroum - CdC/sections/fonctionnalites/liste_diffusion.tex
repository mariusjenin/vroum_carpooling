\section{Liste de diffusions}\label{sec:fonctionnalites-liste_diffusion}

\subsection{Consulter toutes nos listes}\label{subsec:consulter-listes-diffusion}

\begin{itemize}

\item \textit{Sommaire d'identification~:}

    \begin{itemize}
    
    \item Titre~: Consulter toutes nos listes
    
    \item Résumé~: Un utilisateur peut consulter toutes les listes de diffusions qu'il a fait et les sélectionner pour les afficher en détails.
    
    \item Date de création~: 14/01/2021
    
    \item Dernière mise à jour~: 31/01/2021
    
    \item Version~: 1.1
    
    \item Acteurs concernés~: Utilisateurs
    
    \item Responsables~: 
    \begin{itemize}
        \item Antoine Gachenot
    \end{itemize}
    
    \end{itemize}

\item \textit{Description des enchaînements~:}
\begin{itemize}
\item Scénario nominal
    \begin{enumerate}
       \item L'utilisateur sélectionne le bouton "Mes listes d'amis"
       \item Le système affiche toutes les listes d'amis de l'utilisateur sur la page "Mes listes d'amis"
    \end{enumerate}
    \item Scénario alternatif
    \item Scénario exceptionnel
\end{itemize}
\item \textit{Contraintes non fonctionnelles~:}

\item \textit{Pré-conditions~:}

    \begin{itemize}
        \item Être connecté
    \end{itemize}

\item \textit{Post-conditions~:}
    \begin{itemize}
        \item L'utilisateur est sur la page "Mes listes d'amis"
    \end{itemize}
    
    \item \textit{Vues graphiques~:}
    
    Voir images~«~\nameref{fig:Accueil - Connecté}~» et 
«~\nameref{fig:Mes listes d'amis}~»

\end{itemize}

\subsection{Créer une liste}\label{subsec:creer-une-liste}

\begin{itemize}

\item \textit{Sommaire d'identification~:}

    \begin{itemize}
    
    \item Titre~: Créer une liste
    
    \item Résumé~: Un utilisateur peut créer une liste de diffusion pour envoyer des trajets privées à différents utilisateurs. La liste sera identifiée grâce à un nom. 
    
    \item Date de création~: 14/01/2021
    
    \item Dernière mise à jour~: 31/01/2021
    
    \item Version~: 1.1
    
    \item Acteurs concernés~: Utilisateurs
    
    \item Responsables~:
    \begin{itemize}
        \item Antoine Gachenot
        \item Marius Jenin
    \end{itemize}
    
    \end{itemize}

\item \textit{Description des enchaînements~:}
\begin{itemize}
\item Scénario nominal
\begin{enumerate}
       \item L'utilisateur selectionne le bouton "Nouvelle liste"
       \item L'utilisateur saisit le nom de la liste
       \item L'utilisateur valide la modification de la liste en cliquant sur "Valider"
       \item Le système créé la liste et redirige l'utilisateur sur la page "Mes listes d'amis"
       
    \end{enumerate}
    \item Scénario alternatif
    
        \begin{enumerate}[{1{a}.}]\setcounter{enumi}{1}
            \item Deuxième action possible : Ajouter une personne
            \begin{enumerate}[{1.}]\setcounter{enumii}{2}
                \item L'utilisateur saisit l'adresse mail de l'utilisateur qu'il souhaite ajouter
                \item L'utilisateur ajoute l'email qu'il a rentré
                \item Le système vérifie que l'adresse mail correspond bien à un utilisateur 
                \item Le système ajoute cet utilisateur à la liste des emails 
                \item Retour à l'étape 2. du scénario nominal
            \end{enumerate}
        \end{enumerate}
        \begin{enumerate}[{1{b}.}]\setcounter{enumi}{1}
            \item Troisième action possible : Retirer une personne
            \begin{enumerate}[{1.}]\setcounter{enumii}{2}
                \item L'utilisateur clique sur le bouton "Supprimer" de l'utilisateur qu'il souhaite supprimer
                \item Le système supprime l'email de la personne correspondante de la liste des emails affichée
                \item Retour à l'étape 2. du scénario nominal
            \end{enumerate}
        \end{enumerate}
        
        \begin{enumerate}[{1{a}.}]\setcounter{enumi}{3}
            \item Refus de création de liste : une liste possédant le nom donné existe déjà
            \begin{enumerate}[{1.}]\setcounter{enumii}{4}
                \item Le système indique qu'une liste possédant ce nom existe déjà
                \item Retour au cas nominal numéro 2
            \end{enumerate}
        \end{enumerate}
    \item Scénario exceptionnel
\end{itemize}
\item \textit{Contraintes non fonctionnelles~:}

\item \textit{Pré-conditions~:}

    \begin{itemize}
        \item Être connecté
        \item L'utilisateur est sur la page "Mes listes d'amis"
    \end{itemize}

\item \textit{Post-conditions~:}
    \begin{itemize}
        \item La liste de diffusion créée est visible sur la page "Mes listes d'amis"
        \item L'utilisateur est sur la page "Mes listes d'amis"
\end{itemize}

\item \textit{Vues graphiques~:}

Voir images~«~\nameref{fig:Mes listes d'amis}~» et 
«~\nameref{fig:Créer une liste}~»

\end{itemize}

\subsection{Modifier une liste}\label{subsec:modifier-une-liste}

\begin{itemize}

\item \textit{Sommaire d'identification~:}

    \begin{itemize}
    
    \item Titre~: Modifier une liste
    
    \item Résumé~: Un utilisateur peut modifier une liste qu'il a déjà créé. Il peut changer le nom et ajouter ou retirer des personnes.
    
    \item Date de création~: 02/02/2021
    
    \item Dernière mise à jour~: 02/02/2021
    
    \item Version~: 1.1
    
    \item Acteurs concernés~: Utilisateurs
    
    \item Responsables~: 
    \begin{itemize}
        \item Marius Jenin
    \end{itemize}
    
    \end{itemize}


\item \textit{Description des enchaînements~:}
\begin{itemize}
\item Scénario nominal
\begin{enumerate}
       \item L'utilisateur sélectionne la liste qu'il veut modifier
       \item L'utilisateur modifie le nom de sa liste
       \item L'utilisateur valide la modification de la liste en cliquant sur "Valider"
       \item Le système modifie la liste et redirige l'utilisateur sur la page "Mes listes d'amis"
       
    \end{enumerate}
    \item Scénario alternatif
    
        \begin{enumerate}[{1{a}.}]\setcounter{enumi}{1}
            \item Deuxième action possible : Ajouter une personne
            \begin{enumerate}[{1.}]\setcounter{enumii}{2}
                \item L'utilisateur saisit l'adresse mail de l'utilisateur qu'il souhaite ajouter
                \item L'utilisateur ajoute l'email qu'il a rentré
                \item Le système vérifie que l'adresse mail correspond bien à un utilisateur 
                \item Le système ajoute cet utilisateur à la liste des emails 
                \item Retour à l'étape 2. du scénario nominal
            \end{enumerate}
        \end{enumerate}
        \begin{enumerate}[{1{b}.}]\setcounter{enumi}{1}
            \item Troisième action possible : Retirer une personne
            \begin{enumerate}[{1.}]\setcounter{enumii}{2}
                \item L'utilisateur clique sur le bouton "Supprimer" de l'utilisateur qu'il souhaite supprimer
                \item Le système supprime l'email de la personne correspondante de la liste des emails affichés
                \item Retour à l'étape 2. du scénario nominal
            \end{enumerate}
        \end{enumerate}
        
        \begin{enumerate}[{1{a}.}]\setcounter{enumi}{3}
            \item Refus de modification de liste : une liste possédant le nouveau nom donné existe déjà
            \begin{enumerate}[{1.}]\setcounter{enumii}{4}
                \item Le système indique qu'une liste possédant ce nouveau nom existe déjà
                \item Retour au cas nominal numéro 2
            \end{enumerate}
        \end{enumerate}
        
    \item Scénario exceptionnel
\end{itemize}
\item \textit{Contraintes non fonctionnelles~:}

\item \textit{Pré-conditions~:}

    \begin{itemize}
        \item Être connecté
        \item L'utilisateur est sur la page "Mes listes d'amis"
    \end{itemize}

\item \textit{Post-conditions~:}
    \begin{itemize}
        \item La liste de diffusion créée est visible sur la page "Mes listes d'amis"
        \item L'utilisateur est sur la page "Mes listes d'amis"
\end{itemize}

\item \textit{Vues graphiques~:}

Voir images~«~\nameref{fig:Mes listes d'amis}~» et 
«~\nameref{fig:Créer une liste}~»

\end{itemize}

\subsection{Supprimer une liste}\label{subsec:supprimer-une-liste}

\begin{itemize}

\item \textit{Sommaire d'identification~:}

    \begin{itemize}
    
    \item Titre~: Supprimer une liste
    
    \item Résumé~: Un utilisateur peut supprimer une liste de diffusion qui ne lui convient pas.
    
    \item Date de création~: 14/01/2021
    
    \item Dernière mise à jour~: 31/01/2021
    
    \item Version~: 1.1
    
    \item Acteurs concernés~: Utilisateurs
    
    \item Responsables~: 
    \begin{itemize}
        \item Antoine Gachenot
    \end{itemize}
    
    \end{itemize}

\item \textit{Description des enchaînements~:}
\begin{itemize}
\item Scénario nominal
\begin{enumerate}
       \item L'utilisateur clique sur "Supprimer la liste d'amis" correspondant à la liste d'amis
       \item Le système ouvre une fenêtre popup pour confirmer la suppression
       \item L'utilisateur confirme la suppression
       \item Le système supprime cette liste d'amis
    \end{enumerate}
    \item Scénario alternatif
        \begin{enumerate}[{1{a}.}]\setcounter{enumi}{3}
            \item L'utilisateur annule la suppression de l'ami
            \begin{enumerate}[{1.}]\setcounter{enumii}{4}
                \item Fin du scénario alternatif
            \end{enumerate}
        \end{enumerate}
    \item Scénario exceptionnel
\end{itemize}

\item \textit{Contraintes non fonctionnelles~:}

\item \textit{Pré-conditions~:}

    \begin{itemize}
        \item Être connecté
        \item La liste de diffusion existe et est visible sur la page "Mes listes d'amis" de l'utilisateur
        \item L'utilisateur est sur la page "Mes listes d'amis"
    \end{itemize}

\item \textit{Post-conditions~:}
    \begin{itemize}
        \item La liste d'amis n'est plus visible depuis la page "Mes listes d'amis" de l'utilisateur        \item L'utilisateur est sur la page "Mes listes d'amis"
    \end{itemize}
    
    \item \textit{Vues graphiques~:}
    
    Voir image~«~\nameref{fig:Mes listes d'amis}~»


\end{itemize}

% \subsection{Consulter une liste}\label{subsec:consulter-une-liste}

% \begin{itemize}

% \item \textit{Sommaire d'identification~:}

%     \begin{itemize}
    
%     \item Titre~: Consulter une liste
    
%     \item Un utilisateur peut consulter une liste de diffusion spécifique et voir tous les utilisateurs qu'il a ajouté dedans.
    
%     \item Date de création~: 14/01/2021
    
%     \item Dernière mise à jour~: 31/01/2021
    
%     \item Version~: 1.1
    
%     \item Acteurs concernés~: Utilisateurs
    
%     \item Responsables~: 
%     \begin{itemize}
%         \item Antoine Gachenot
%     \end{itemize}
    
%     \end{itemize}

% \item \textit{Description des enchaînements~:}
% \begin{itemize}
% \item Scénario nominal
% \begin{enumerate}
%       \item L'utilisateur clique sur la liste d'amis qu'il souhaite consulter
%       \item Le système redirige l'utilisateur sur la page de la liste d'amis
%     \end{enumerate}
%     \item Scénario alternatif
%     \item Scénario exceptionnel
% \end{itemize}

% \item \textit{Contraintes non fonctionnelles~:}

% \item \textit{Pré-conditions~:}

%     \begin{itemize}
%         \item Être connecté
%         \item La liste de diffusion existe et est visible sur la page "Mes listes d’amis" de l’utilisateur
%         \item L'utilisateur est sur la page "Mes listes d'amis"
%     \end{itemize}

% \item \textit{Post-conditions~:}
%     \begin{itemize}
%         \item L'utilisateur est sur la page correspondante à la liste d'amis qu'il souhaite consulter
%     \end{itemize}
    
%     \item \textit{Vues graphiques~:}
    
%     Voir images~«~\nameref{fig:Mes listes d'amis}~» et 
% «~\nameref{fig:Créer une liste}~»

% \end{itemize}
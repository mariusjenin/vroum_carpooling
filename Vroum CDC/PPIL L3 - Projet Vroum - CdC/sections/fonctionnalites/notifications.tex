\section{Notifications}\label{sec:fonctionnalites-notifications}

Les notifications permettent aux utilisateurs qui ne lisent pas leurs mails (ou aux utilisateurs qui ne souhaitent pas recevoir de mails) d'être quand même avertis de changements pouvant les concerner.

\subsection{Consulter nos notifications}\label{subsec:consulter-nos-notifications}

\begin{itemize}

\item \textit{Sommaire d'identification~:}

    \begin{itemize}
    
    \item Titre~: Consulter nos notifications
    
    \item Résumé~: Un utilisateur peut accumuler les notifications s'il ne s'est pas connecté depuis un bout de temps.
    Il est donc nécessaire de pouvoir consulter toutes nos notifications, afin de se tenir à jour des nouvelles nous concernant.
    
    \item Date de création~: 14/01/2021
    
    \item Dernière mise à jour~: 28/01/2021
    
    \item Version~: 1.1
    
    \item Acteurs concernés~: Utilisateurs
    
    \item Responsables~:

        \begin{itemize}
            \item Ghilain Bergeron
        \end{itemize}
    
    \end{itemize}

\item \textit{Description des enchaînements~:}

    \begin{itemize}
        \item Scénario nominal
        \begin{enumerate}
            \item L'utilisateur accède à sa boîte de réception
            \item Le système récupère et affiche les expéditeurs des notifications reçues
            \item L'utilisateur clique sur un expéditeur pour avoir plus de détails sur les notifications reçues
        \end{enumerate}
        \item Scénario alternatif
        \item Scénario exceptionnel
    \end{itemize}

\item \textit{Contraintes non fonctionnelles~:}

    \begin{itemize}
        \item Temps de réponse inférieur à 10 secondes
    \end{itemize}

\item \textit{Pré-conditions~:}

    \begin{itemize}
        \item L'utilisateur est connecté
        \item L'utilisateur se trouve sur sa page d'accueil
    \end{itemize}

\item \textit{Post-conditions~:}

    \begin{itemize}
        \item Toutes les notifications non lues sont marquées comme lues lors de la sortie de la boîte de réception.
    \end{itemize}
    
\item \textit{Vues graphiques~:}

Voir images~«~\nameref{fig:Mes notifications - Utilisateurs}~» et «~\nameref{fig:Mes notifications - Notifications}~»

\end{itemize}

\subsection{Supprimer une notification}\label{subsec:supprimer-une-notification}

\begin{itemize}

\item \textit{Sommaire d'identification~:}

    \begin{itemize}
    
    \item Titre~: Supprimer une notification
    
    \item Résumé~: Les notifications peuvent s'accumuler rapidement, et ainsi prendre de la place inutilement.
          Il est donc nécessaire de pouvoir supprimer de vieilles notifications (ou des notifications lues) afin de ne pas trop polluer sa boîte de réception.
    
    \item Date de création~: 14/01/2021
    
    \item Dernière mise à jour~: 28/01/2021
    
    \item Version~: 1.1
    
    \item Acteurs concernés~: Utilisateurs
    
    \item Responsables~:
    
        \begin{itemize}
            \item Ghilain Bergeron
        \end{itemize}
    
    \end{itemize}

\item \textit{Description des enchaînements~:}

    \begin{itemize}
        \item Scénario nominal
        \begin{enumerate}
            \item L'utilisateur accède à sa boîte de réception
            \item Le système récupère et affiche les expéditeurs des notifications reçues
            \item L'utilisateur clique sur un expéditeur pour avoir plus de détails sur les notifications
            \item L'utilisateur choisit une notification à supprimer
            \item Le système supprime la notification sélectionnée par l'utilisateur
        \end{enumerate}
        \item Scénario alternatif
        \item Scénario exceptionnel
    \end{itemize}

\item \textit{Contraintes non fonctionnelles~:}

    \begin{itemize}
        \item Temps de réponse inférieur à 1 seconde par notification supprimée
    \end{itemize}

\item \textit{Pré-conditions~:}

    \begin{itemize}
        \item L'utilisateur est connecté
        \item L'utilisateur se trouve sur la page de sa boîte de réception
    \end{itemize}

\item \textit{Post-conditions~:}

    \begin{itemize}
        \item Les notifications sélectionnées ne figurent plus dans la boîte de réception
    \end{itemize}
    
\item \textit{Vues graphiques~:}

Voir image~«~\nameref{fig:Mes notifications - Notifications}~»

\end{itemize}